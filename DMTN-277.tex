\documentclass[DM,authoryear,toc]{lsstdoc}
\input{meta}

% Package imports go here.

% Local commands go here.

%If you want glossaries
%\input{aglossary.tex}
%\makeglossaries

\title{The Monster: a southern reference catalog with synthetic ugrizy fluxes for the Vera C. Rubin observatory}

% Optional subtitle
% \setDocSubtitle{A subtitle}

\author{%
Peter Ferguson
}

\setDocRef{DMTN-277}
\setDocUpstreamLocation{\url{https://github.com/lsst-dm/dmtn-277}}

\date{\vcsDate}

% Optional: name of the document's curator
% \setDocCurator{The Curator of this Document}

\setDocAbstract{%
In order to facilitate bootstrap photometric calibrations of early Rubin Observatory data we have created the Monster. This reference catalog uses a ranked order set of other reference catalogs to generate synthetic ugrizy fluxes that can be used calibrate images processed with the LSST science pipelines. This document describes the methodology used to create the Monster, documents the input reference catalogs, and performs basic data validation the Monster reference catalog.
}

% Change history defined here.
% Order: oldest first.
% Fields: VERSION, DATE, DESCRIPTION, OWNER NAME.
% See LPM-51 for version number policy.
\setDocChangeRecord{%
  \addtohist{1}{YYYY-MM-DD}{Unreleased.}{Peter Ferguson}
}


\begin{document}

% Create the title page.
\maketitle
% Frequently for a technote we do not want a title page  uncomment this to remove the title page and changelog.
% use \mkshorttitle to remove the extra pages

% ADD CONTENT HERE
% You can also use the \input command to include several content files.

\appendix
% Include all the relevant bib files.
% https://lsst-texmf.lsst.io/lsstdoc.html#bibliographies
\section{References} \label{sec:bib}
\renewcommand{\refname}{} % Suppress default Bibliography section
\bibliography{local,lsst,lsst-dm,refs_ads,refs,books}

% Make sure lsst-texmf/bin/generateAcronyms.py is in your path
\section{Acronyms} \label{sec:acronyms}
\addtocounter{table}{-1}
\begin{longtable}{p{0.145\textwidth}p{0.8\textwidth}}\hline
\textbf{Acronym} & \textbf{Description}  \\\hline

ATLAS & A Toroidal LHC Apparatus \\\hline
CCD & Charge-Coupled Device \\\hline
DEC & Declination \\\hline
DECam & Dark Energy Camera \\\hline
DES & Dark Energy Survey \\\hline
DM & Data Management \\\hline
DMTN & DM Technical Note \\\hline
DR3 & Data Release 3 \\\hline
ESO & European Southern Observatory \\\hline
FGCM & Forward Global Calibration Model \\\hline
GB & Gigabyte \\\hline
HTM & Hierarchical Triangular Mesh \\\hline
LATISS & LSST Atmospheric Transmission Imager and Slitless Spectrograph \\\hline
LSE & LSST Systems Engineering (Document Handle) \\\hline
LSST & Legacy Survey of Space and Time (formerly Large Synoptic Survey Telescope) \\\hline
OSS & Observatory System Specifications; LSE-30 \\\hline
PS1 & Pan-STARRS 1 survey \\\hline
RA & Risk Assessment \\\hline
RMS & Root-Mean-Square \\\hline
SDSS & Sloan Digital Sky Survey \\\hline
VST & VLT Survey Telescope \\\hline
\end{longtable}

% If you want glossary uncomment below -- comment out the two lines above
%\printglossaries





\end{document}
